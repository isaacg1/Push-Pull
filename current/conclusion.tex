\section{Conclusion}
In this paper, we proved hardness results about variations of block-pushing puzzles in which the robot can also pull blocks. Along the way, we analyzed the complexity of two new, simple gadgets, creating useful new toolsets with which to attack hardness of future puzzles. The results themselves are obviously of interest to game and puzzle enthusiasts, but we also hope the analysis leads to a better understanding of motion-planning problems more generally and that the techniques we developed allow us to better understand the complexity of related problems.

This work leads to many open questions to pursue in future research. For Push-Pull block puzzles, we leave several NP vs.\ PSPACE gaps, a feature shared with many block-pushing puzzles. One would hope to directly improve upon the results here to show tight hardness results for 2D and 3D push-pull block puzzles. One might also wonder if the gadgets used, or the introduction of thin walls, might lead to stronger results for other block-pushing puzzles. We also leave open the question of push-pull block puzzles without fixed blocks or walls. In this setting, even a single $3\times3$ area of clear space allows the robot to reach any point, making gadget creation challenging.

There are also interesting questions with respect to the abstract gadgets introduced in our proof. We are currently studying the complexity of smaller toggles and toggle-lock systems. It would also be interesting to know whether Set-Verify gadgets sufficient for PSPACE-hardness or if they can build full crossover gadgets. Also, there are also many variations within the framework of connected blocks with traversibility which changes with passage through the gadget. Are any other gadgets within this framework useful for capturing salient features of motion planning problems? Finally, there is the question of whether other computational complexity problems can make use of these gadgets to prove new results.
