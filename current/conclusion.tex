\subsection{Implied Hardness Results}
\xxx{lets look up some games with push-pull block puzzles. I had ruble finding them (many were slight variations like railed versions or with the ability to climb on blocks). Actually perhaps we should pick up some corollaries that some of these (push pull on rails or with climbing or with fixed moves or some such) are hard.}

\section{Conclusion}
In this thesis we proved a number of hardness results about single-agent games. The results themselves are obviously of interest to game and puzzle enthusiasts. We also hope the study of motion planning in environments with dynamic topologies leads to new insights. What we consider most interesting are the techniques developed to solve these problems. Looking back, many of the Portal proofs could have been phrased in terms of the style of agent-gadget diagrams seen in the Push-Pull blocks section. Adding new, simple gadgets to this collection of abstractions gives us powerful new tools with which to attack future problems. We also believe the decomposition of games into individual mechanics will be an important tactic for understanding games of increasing complexity.

In this paper we proved a few hardness results about different varieties of block pushing puzzles. Along the way we analyzed the complexity of two new, simple gadgets which may be useful new tools with which to attack future problems. The results themselves are obviously of interest to game and puzzle enthusiasts, but we also hope the analysis leads to a better understanding of motion planning problems more generally and the techniques allow us to better understand the complexity of problems.

\section{Open Questions}
This work leads to many open questions to pursue in future research. For Push-Pull block puzzles, we leave a number of NP to PSPACE gaps, as is common for many other variations of the problem. One would hope to directly improve upon the results here to show tight hardness results for 2D and 3D push-pull block puzzles. One might also wonder if the gadgets used, or the introduction of thin walls might lead to stronger results for other block pushing puzzles. We also leave open the question of push-pull block puzzles without fixed walls. Notice that a single $3\times3$ block of clear space allows the robot to reach any point, making gadget creation challenging.

There are also interesting questions with regard to the abstract gadgets used in the proof. Are 2-toggles or 3-toggles sufficient to prove NP-hardness or PSPACE-hardness? Can one construct crossovers out of toggles? Are Set-Verify gadgets sufficient for PSPACE-hardness? Is this model of abstraction useful for capturing more agent-based games and puzzles? Finally, one would hope to use these techniques to show hardness for other problems. The toggle and Set-Verify abstractions add new, simple gadgets for proving hardness. 
%
%\section*{Acknowledgments}
%I'd like to thank my advisor, Erik Demaine, for his support and insight throughout this project. Isaac Grossof made major contributions to the proofs related to block pushing puzzles; those results would not have occurred without his collaboration. The proof of hardness of Portal with turrets, as well as how to extend the results, grew out of ideas from a correspondence with Joshua Lockhart. The terminology use in this thesis would not have been nearly as consistent without Elizabeth Krueger's editing help. I'd like to thank all of my parents who have been essential in allowing me to get to this point.
%
%Finally, I'd like to dedicate this thesis to Andrea Lincoln who has been my best friend through all of the years of this project. Your intellectual collaboration and emotional support have been invaluable.
%Erik, Isaac, Joshua Lockhart, Parents