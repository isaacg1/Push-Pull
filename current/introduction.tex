
\section{Introduction} 
%Block-pushing puzzles are a common puzzle type with one of the best known example being Sokoban. They have found their way into several popular video games including The Legend of Zelda, Pok\'emon, and Tomb Raider. Block-pushing puzzles are also a recreational embodiment of motion planning problems with movable obstacles. Because motion planning is such an important and computationally difficult problem, it can be useful to look at simplified models to try to get a better understanding of the larger problem.

Block-pushing puzzles are a common puzzle type with one of the best known example being \emph{Sokoban}. Puzzles with the ability to push and pull blocks have found their way into several popular video games including \emph{The Legend of Zelda} searies, \emph{Starfox Adventures}, \emph{Half-Life} and \emph{Tomb Raider}. Block-pushing puzzles are also an abstraction of motion planning problems with movable obstacles. In addition to these games, one could imagine real-world scenarios, like that of a forklift in a warehouse, bearing similarity. Since motion planning is such an important and computationally difficult problem, it can be useful to look at simplified models to try to get a better understanding of the larger problem.

%Push-pull block puzzles form a natural abstraction of puzzles in several video games. \emph{Pukoban} directly implements this type of puzzle. Other games also include block puzzles with the ability to push and pull blocks. The \emph{Legend of Zelda} series introduces push-pull block puzzles in \emph{A Link to the Past}, and this mechanic recurs in most subsequent games. Other games include \emph{Starfox Adventures}, \emph{Catherine}, and the original \emph{Tomb Raider}. Even more general block manipulation shows up in \emph{Half-Life 1} and \emph{Half-Life 2}, and in \emph{Portal} and \emph{Portal 2}.  In addition to representing a natural abstraction of these real video games,
%the push-pull model is a natural abstract form of many real-world motion planning problems, as one could imagine the constraints of a forklift in a warehouse bearing similarity.


%We also give a new result showing that block pushing puzzles, like Sokaban, which also include the ability to pull blocks are NP-hard. This result also proves a number of other games, which have embedded push-pull block puzzles, are NP-hard. Games have been and continue to be a source of interesting problems for mathematics and computer science, and game playing continues to be a benchmark in the field of AI. The study of economic games has become a large field with applications in online auctions\cite{auctions}, network analysis\cite{Internet}, and voting\cite{colman2013game}. Combinatorial games have led to algebraic insights such as Conway's Surreal Numbers\cite{Surreal01} and the Sprague-Grundy Theorem\cite{Sprague35, Grundy39}. More recently, games have been studied from an algorithmic perspective and can be seen as models of computation in examples like Constraint Logic \cite{GPCBook09} and Conway's Game of Life\cite{LifeTuring01}. For a survey of algorithmic combinatorial game theory, see Demaine and Hearn's paper\cite{AlgGameTheory_GONC3}.
%
%There has been a surge of recent interest in the computational complexity of video games and puzzles. Examples of previous work in this area includes the NP-completeness of Minesweeper\cite{Minesweeper00}, Clickomania\cite{ClickomaniaGameTheory2000}, and Tetris\cite{Tetris03}, as well as hardness for Lemmings\cite{Lemmings04, viglietta2015lemmings}.    Some surveys of the computational complexity of puzzles, including video games, include Demaine and Hearn\cite{AlgGameTheory_GONC3}, and Kendall, Parkes and Spoerer\cite{NPPuzzles08}. Recent work has moved from puzzles to classic arcade games\cite{HardGames12}, Nintendo games\cite{NintendoFun2014}, 2D platform games\cite{Forisek10}, and others\cite{DBLP:journals/corr/Walsh14, floodIt, DBLP:journals/corr/abs-1203-1633}. 

A significant amount of research has gone into characterizing the complexity of block sliding puzzles. This includes PSPACE-completeness for well-known puzzles like sliding-block puzzles\cite{hearn2005pspace}, Sokoban \cite{Sokoban98, DZ96}, the 15-puzzle \cite{15Puzzle}, and Rush Hour \cite{RushHour02}. Block pushing puzzles are a type of block sliding puzzle in which the blocks are moved by a small robot within the puzzles. This type of block sliding puzzle has gathered a significant amount of study. Variations include Sokoban\cite{Sokoban98, DZ96}, where blocks must reach specific targets, versions where multiple blocks can be pushed\cite{Push100, Push*00, Push2F02}, versions where blocks continue to slide after being pushed\cite{PushPushk04, Push*00}, and versions where the robot can pull blocks\cite{Pull10}. The problem of motion planning in an environment where blocks may be pushed and pulled is modeled in a general form in Gordon Wilfong's Motion Planning in the Presence of Movable Obstacles\cite{PushPull91}. There he shows a polynomial time algorithm for motion planning with one movable object, NP-hardness for the general planning problem, and PSPACE-hardness for the storage problem. Table~\ref{BlocksTable} gives a summary of results on block pushing puzzles. 

%We are particularly interested in the push-pull block model because it is fully reversible, meaning any sequence of moves in the puzzle can be undone. Reversibility is fundamentally linked to quantum computation and the thermodynamics of computation. 

%In this paper, we prove results about push-pull block puzzles. Push-pull block puzzles can be seen as a simplified model of robotic forklifts operating in a warehouse. Complexity results in these very simplified models help us understand what aspects of problems makes them hard and allows us to start differentiating between the complexity that arises from the combinatorics vs the geometry of path planning problems. Similarly, path planning in dynamic graphs is a complicated problem which occurs every day from traffic to packet routing.

We add several new results showing that certain block pushing puzzles, which include the ability to push and pull blocks, are NP-hard or PSPACE-complete. The push-pull block puzzle is instantiated in the game Pukoban and heuristics for solving it have been studied \cite{zubaranagent}, but its computational complexity was left as an open question.
We introduce \emph{thin walls}, which prevent motion between two adjacent empty squares. We prove that all path planning problems in 2D with thin walls or in 3D, in which the robot can push $k$ blocks and pull $l$ blocks for all $k,l \in \mathbb{Z}^+$ are NP-hard. Our results are shown in the last four lines of Table~\ref{BlocksTable}. To prove these results, we introduce two new abstract gadgets, the set-verify and the 4-toggle, and prove hardness results for questions about the legal state transitions of these gadgets. %As with many of these problems, closing the gap between NP and PSPACE remains open.

%Push-pull block puzzles form a natural abstraction of puzzles in several video games. \emph{Pukoban} directly implements this type of puzzle. Other games also include block puzzles with the ability to push and pull blocks. The \emph{Legend of Zelda} series introduces push-pull block puzzles in \emph{A Link to the Past}, and this mechanic recurs in most subsequent games. Other games include \emph{Starfox Adventures}, \emph{Catherine}, and the original \emph{Tomb Raider}. Even more general block manipulation shows up in \emph{Half-Life 1} and \emph{Half-Life 2}, and in \emph{Portal} and \emph{Portal 2}.  In addition to representing a natural abstraction of these real video games,
%the push-pull model is a natural abstract form of many real-world motion planning problems, as one could imagine the constraints of a forklift in a warehouse bearing similarity.

%% Starfox Adventures (push-pull-sideways), Catherine (push-pull-climbing, central mechanic)
%%Ocarina of Time: https://www.youtube.com/watch?v=nRM8p2re_cY 
%%Link to the Past: http://www.thonky.com/zelda-link-to-the-past/palace-of-darkness 
%%Minish Cap: http://www.gamespot.com/articles/the-legend-of-zelda-the-minish-cap-walkthrough/1100-6116391/
%%Original Tomb Raider: http://superadventuresingaming.blogspot.com/2014/11/tomb-raider-ms-dos-part-2.html
\begin{table}
\centering
\centerline{
\begin{tabular}{|l|l|l|l|l|l|l|l|l|}
\hline
\emph{Name} & \emph{Push} & \emph{Pull} & \emph{Blocks} & \emph{Fixed?} & \emph{Path?} & \emph{Sliding} & \emph{Complexity} \\ \hline
\hline
Push-$k$ & $k$ & 0 & Unit & No & Path & Min & NP-hard\cite{Push100} \\ \hline
Push-$*$ & $*$ & 0 & Unit & No & Path  & Min & NP-hard\cite{Push*00} \\ \hline
PushPush-$k$ & $k$ & 0 & Unit & No & Path  & Max & PSPACE-c.\cite{PushPushk04} \\ \hline
PushPush-$*$ & $*$ & 0 & Unit & No & Path  & Max & NP-hard\cite{Push*00} \\ \hline
%Push-$k$X & $k$ & $0$ & Unit & No & No-Cross  & Min & NP-c.\cite{non-crossing01} \\ \hline
%Push-$*$X & $*$ & $0$ & Unit & No & No-Cross  & Min & NP-c.\cite{non-crossing01} \\ \hline
Push-$1$F & $1$ & $0$ & Unit & Yes & Path  & Min &  NP-hard \cite{DO92} \\ \hline
Push-$k$F & $k\geq 2$ & $0$ & Unit & Yes & Path  & Min & PSPACE-c.\cite{Push2F02} \\ \hline
Push-$*$F & $*$ & $0$ & Unit & Yes & Path  & Min & PSPACE-c.\cite{Push2F02} \\ \hline
Sokoban & $1$ & $0$ & Unit & Yes & Storage  & Min & PSPACE-c.\cite{Sokoban98} \\ \hline
%Sokoban$^+$ & $k\geq 2$ & $1$ & 2x1 & Yes & Storage  & Min & PSPACE-c.\cite{DZ96} \\ \hline
Sokoban$(k,1)$ & $k\geq 5$ & $1$ & Unit & Yes & Storage  & Min & NP-hard\cite{DZ96} \\ \hline
%Motion Planning & $k$ & $1$  & L & Yes & Storage  & Min & NP-hard\cite{PushPull91}\\ \hline
Pull-$1$ & $0$ & $1$ & Unit & No & Storage  & Min & NP-hard\cite{Pull10} \\ \hline
Pull-$k$F & $0$ & $k$ & Unit & Yes & Storage  & Min &  NP-hard\cite{Pull10} \\ \hline
PullPull-$k$F & $0$ & $k$ & Unit & Yes & Storage  & Max  & NP-hard\cite{Pull10} \\ \hline
%Push-$1$G & $1$ & $0$ & Unit & Yes &  Path  & Min  & NP-hard\cite{Gravity} \\ \hline
\textbf{Push-$k$ Pull-$l$W} & $k$ & $l$ & Unit & Wall & Path  & Min & \textbf{NP-hard}\ (\S  \ref{2DNPhard}) \\ \hline
\textbf{3D Push-$k$ Pull-$l$F} & $k$ & $l$ & Unit & Yes & Path & Min &  \textbf{NP-hard}\ (\S  \ref{3DNPhard}) \\ \hline
\textbf{3D Push-$1$ Pull-$1$W} & $1$ & $1$ & Unit & Wall & Path & Min &  \textbf{PSPACE-c.}\ (\S  \ref{3DPSPACE}) \\ \hline
\textbf{3D Push-$k$ Pull-$k$F} & $k > 1$ & $k >1$ & Unit & Yes & Path & Min &  \textbf{PSPACE-c.}\ (\S  \ref{3DPSPACE}) \\ \hline
\end{tabular}
}
\caption{Summary of past and new results on block pushing and/or pulling. Here $k$ and $l$ are positive integers; $*$ refers to an unlimited number of blocks. F means fixed blocks are included; W means thin walls are included.} %X means the robot cannot step on the same square more than once. G means the blocks are subject to gravity.
\label{BlocksTable}
\end{table}

%We are particularly interested in the push-pull block model because it is fully reversible, meaning any sequence of moves in the puzzle can be undone. Reversibility is fundamentally linked to quantum computation and the thermodynamics of computation. 

%In this paper, we prove results about push-pull block puzzles. Push-pull block puzzles can be seen as a simplified model of robotic forklifts operating in a warehouse. Complexity results in these very simplified models help us understand what aspects of problems makes them hard and allows us to start differentiating between the complexity that arises from the combinatorics vs the geometry of path planning problems. Similarly, path planning in dynamic graphs is a complicated problem which occurs every day from traffic to packet routing.

We add several new results showing that certain block pushing puzzles, which include the ability to push and pull blocks, are NP-hard or PSPACE-complete. The push-pull block puzzle is instantiated in the game Pukoban and heuristics for solving it have been studied \cite{zubaranagent}, but its computational complexity was left as an open question.
We introduce \emph{thin walls}, which prevent motion between two adjacent empty squares. We prove that all path planning problems in 2D with thin walls or in 3D, in which the robot can push $k$ blocks and pull $l$ blocks for all $k,l \in \mathbb{Z}^+$ are NP-hard. We also show that path planning problems where the robot can push and pull $k$ blocks are PSPACE-complete, with thin walls needed only for $k=1$. Our results are shown in the last four lines of Table~\ref{BlocksTable}. To prove these results, we introduce two new abstract gadgets, the set-verify and the 4-toggle, and prove hardness results for questions about the legal state transitions of these gadgets. %As with many of these problems, closing the gap between NP and PSPACE remains open.

%Push-pull block puzzles form a natural abstraction of puzzles in several video games. \emph{Pukoban} directly implements this type of puzzle. Other games also include block puzzles with the ability to push and pull blocks. The \emph{Legend of Zelda} series introduces push-pull block puzzles in \emph{A Link to the Past}, and this mechanic recurs in most subsequent games. Other games include \emph{Starfox Adventures}, \emph{Catherine}, and the original \emph{Tomb Raider}. Even more general block manipulation shows up in \emph{Half-Life 1} and \emph{Half-Life 2}, and in \emph{Portal} and \emph{Portal 2}.  In addition to representing a natural abstraction of these real video games,
%the push-pull model is a natural abstract form of many real-world motion planning problems, as one could imagine the constraints of a forklift in a warehouse bearing similarity.

2D Push-$k$ Pull-$j$ is defined as follows: There is a square lattice of cells. Each cell is connected to its orthogonal neighbors. Cells may either be empty, hold a movable block, or hold a fixed block. Additionally, in settings that allow thin walls, edges between cells may be omitted. There is also a robot on a cell. The robot may move from it's current cell to an unoccupied adjacent cell. The robot may also push up to $k$ movable blocks arranged in a straight line one cell forward, as long as there is an open cell with no wall in that direction. Likewise, the robot may pull up to $j$ movable blocks as long as there are no walls in the way and there is an open cell behind the robot. The goal of the puzzle is for the robot to reach a specified goal cell. Given such a description, is there a legal path for the robot from it's starting cell to the goal cell? The 3D problem is defined analogously on a cubic lattice.

