
\section{Introduction} 
Block pushing puzzles are a common puzzle type with perhaps the best known example being Sokoban. They have found their way into several popular video games including The Legend of Zelda, Pokemon, and Tomb Raider. Block pushing puzzles are a recreational embodiment of motion planning problems with movable obstacles. Since motion planning is such an important and computationally difficult problem, it can be useful to look at simplified models to try to get a better understanding of the larger problem. The push-pull model itself also doesn't seem entirely abstract, as one could imagine the constraints of a forklift in a warehouse bearing similarity.

%We also give a new result showing that block pushing puzzles, like Sokaban, which also include the ability to pull blocks are NP-hard. This result also proves a number of other games, which have embedded push-pull block puzzles, are NP-hard. Games have been and continue to be a source of interesting problems for mathematics and computer science, and game playing continues to be a benchmark in the field of AI. The study of economic games has become a large field with applications in online auctions\cite{auctions}, network analysis\cite{Internet}, and voting\cite{colman2013game}. Combinatorial games have led to algebraic insights such as Conway's Surreal Numbers\cite{Surreal01} and the Sprague-Grundy Theorem\cite{Sprague35, Grundy39}. More recently, games have been studied from an algorithmic perspective and can be seen as models of computation in examples like Constraint Logic \cite{GPCBook09} and Conway's Game of Life\cite{LifeTuring01}. For a survey of algorithmic combinatorial game theory, see Demaine and Hearn's paper\cite{AlgGameTheory_GONC3}.
%
%There has been a surge of recent interest in the computational complexity of video games and puzzles. Examples of previous work in this area includes the NP-completeness of Minesweeper\cite{Minesweeper00}, Clickomania\cite{ClickomaniaGameTheory2000}, and Tetris\cite{Tetris03}, as well as hardness for Lemmings\cite{Lemmings04, viglietta2015lemmings}.    Some surveys of the computational complexity of puzzles, including video games, include Demaine and Hearn\cite{AlgGameTheory_GONC3}, and Kendall, Parkes and Spoerer\cite{NPPuzzles08}. Recent work has moved from puzzles to classic arcade games\cite{HardGames12}, Nintendo games\cite{NintendoFun2014}, 2D platform games\cite{Forisek10}, and others\cite{DBLP:journals/corr/Walsh14, floodIt, DBLP:journals/corr/abs-1203-1633}. 

A significant amount of research has gone into characterizing the complexity of block sliding puzzles. This includes PSPACE-completeness for well-known puzzles like sliding-block puzzles\cite{hearn2005pspace}, Sokoban \cite{Sokoban98, DZ96}, the 15-puzzle \cite{15Puzzle}, and Rush Hour \cite{RushHour02}. Block pushing puzzles are a type of block sliding puzzle in which the blocks are moved by a small robot within the puzzles. This type of block sliding puzzle has gathered a significant amount of study. Variations include Sokoban\cite{Sokoban98, DZ96}, where blocks must reach specific targets, versions where multiple blocks can be pushed\cite{Push100, Push*00, Push2F02}, versions where blocks continue to slide after being pushed\cite{PushPushk04, Push*00}, versions where the robot can pull blocks\cite{Pull10}, and versions where the blocks are subject to gravity\cite{Gravity}. The problem of motion planning in an environment where blocks may be pushed and pulled is modeled in a general form in Gordon Wilfong's Motion Planning in the Presence of Movable Obstacles\cite{PushPull91}. There he shows a polynomial time algorithm for motion planning with one movable object, NP-hardness for the general planning problem, and PSPACE-hardness for the storage problem. Table~\ref{BlocksTable} gives a summary of results on block pushing puzzles. 

%We are particularly interested in the push-pull block model because it is fully reversible, meaning any sequence of moves in the puzzle can be undone. Reversibility is fundamentally linked to quantum computation and the thermodynamics of computation. 

%In this paper, we prove results about push-pull block puzzles. Push-pull block puzzles can be seen as a simplified model of robotic forklifts operating in a warehouse. Complexity results in these very simplified models help us understand what aspects of problems makes them hard and allows us to start differentiating between the complexity that arises from the combinatorics vs the geometry of path planning problems. Similarly, path planning in dynamic graphs is a complicated problem which occurs every day from traffic to packet routing.

We add several new results showing that certain block pushing puzzles, which include the ability to push and pull blocks, are NP-hard or PSPACE-complete. We introduce \emph{thin walls}, which prevent motion between two adjacent empty squares. We prove that all path planning problems in 2D with thin walls or in 3D, in which the robot can push $k$ blocks and pull $l$ blocks for all $k,l \in \mathbb{Z}^+$ are NP-hard. Our results are shown in the last four lines of Table~\ref{BlocksTable}. To prove these results we introduce two new abstract gadgets, the set-verify and the 4-toggle, and prove hardness results for questions about the legal state transitions of these gadgets. %As with many of these problems, closing the gap between NP and PSPACE remains open.


\begin{table}
\centering
\centerline{
\begin{tabular}{|l|l|l|l|l|l|l|l|l|}
\hline
\emph{Name} & \emph{Push} & \emph{Pull} & \emph{Blocks} & \emph{Fixed?} & \emph{Path?} & \emph{Sliding} & \emph{Complexity} \\ \hline
\hline
Push-$k$ & $k$ & 0 & Unit & No & Path & min & NP-hard\cite{Push100} \\ \hline
Push-$*$ & $*$ & 0 & Unit & No & Path  & min & NP-hard\cite{Push*00} \\ \hline
PushPush-$k$ & $k$ & 0 & Unit & No & Path  & Max & PSPACE-c.\cite{PushPushk04} \\ \hline
PushPush-$*$ & $*$ & 0 & Unit & No & Path  & Max & NP-hard\cite{Push*00} \\ \hline
Push-$k$X & $k$ & $0$ & Unit & No & No-Cross  & min & NP-c.\cite{non-crossing01} \\ \hline
Push-$*$X & $*$ & $0$ & Unit & No & No-Cross  & min & NP-c.\cite{non-crossing01} \\ \hline
Push-$1$F & $1$ & $0$ & Unit & Yes & Path  & min &  NP-hard \cite{DO92} \\ \hline
Push-$k$F & $k\geq 2$ & $0$ & Unit & Yes & Path  & min & PSPACE-c.\cite{Push2F02} \\ \hline
Push-$*$F & $*$ & $0$ & Unit & Yes & Path  & min & PSPACE-c.\cite{Push2F02} \\ \hline
Sokoban & $1$ & $0$ & Unit & Yes & Storage  & min & PSPACE-c.\cite{Sokoban98} \\ \hline
Sokoban$^+$ & $k\geq 2$ & $1$ & 2x1 & Yes & Storage  & min & PSPACE-c.\cite{DZ96} \\ \hline
Sokoban$(k,1)$ & $k\geq 5$ & $1$ & Unit & Yes & Storage  & min & NP-hard\cite{DZ96} \\ \hline
Motion Planning & $k$ & $1$  & L & Yes & Storage  & min & NP-hard\cite{PushPull91}\\ \hline
Pull-$1$ & $0$ & $1$ & Unit & No & Storage  & min & NP-hard\cite{Pull10} \\ \hline
Pull-$k$F & $0$ & $k$ & Unit & Yes & Storage  & min &  NP-hard\cite{Pull10} \\ \hline
PullPull-$k$F & $0$ & $k$ & Unit & Yes & Storage  & Max  & NP-hard\cite{Pull10} \\ \hline
Push-$1$G & $1$ & $0$ & Unit & Yes &  Path  & min  & NP-hard\cite{Gravity} \\ \hline
\textbf{Push-$k$ Pull-$l$W} & $k$ & $l$ & Unit & Wall & Path  & min & \textbf{NP-hard}\ (\S  \ref{2DNPhard}) \\ \hline
\textbf{3D Push-$k$ Pull-$l$F} & $k$ & $l$ & Unit & Yes & Path & min &  \textbf{NP-hard}\ (\S  \ref{3DNPhard}) \\ \hline
\textbf{3D Push-$1$ Pull-$1$W} & $1$ & $1$ & Unit & Wall & Path & min &  \textbf{PSPACE-c.}\ (\S  \ref{3DPSPACE}) \\ \hline
\textbf{3D Push-$k$ Pull-$k$F} & $k > 1$ & $k >1$ & Unit & Yes & Path & min &  \textbf{PSPACE-c.}\ (\S  \ref{3DPSPACE}) \\ \hline
\end{tabular}
}
\caption{Summary of Past and New Block Pushing Puzzle Results. $k$ and $l$ are positive integers; $*$ refers to an unlimited number of blocks. F means fixed blocks are included. W means thin walls are included. X means the robot cannot step on the same square more than once. G means the blocks are subject to gravity.}
\label{BlocksTable}
\end{table}

