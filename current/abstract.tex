\begin{abstract}
This paper proves that push-pull block puzzles in 3D are PSPACE-complete to solve, and push-pull block puzzles in 2D with thin walls are NP-hard to solve, settling an open question \cite{zubaranagent}. Push-pull block puzzles are a type of recreational motion planning problem, similar to Sokoban, that involve moving a `robot' on a square grid with $1 \times 1$ obstacles. The obstacles cannot be traversed by the robot, but some can be pushed and pulled by the robot into adjacent squares. Thin walls prevent movement between two adjacent squares. This work follows in a long line of algorithms and complexity work on similar problems \cite{PushPull91,Push100,Push*00,PushPushk04,non-crossing01,DO92,Push2F02,Sokoban98,DZ96,Pull10}. The 2D push-pull block puzzle shows up in the video games \emph{Pukoban} as well as \emph{The Legend of Zelda: A Link to the Past}, giving another proof of hardness for the latter \cite{NintendoFun2014}. This variant of block-pushing puzzles is of particular interest because it is fully reversible, meaning that any action (e.g., push or pull) can be inverted by another valid action (e.g., pull or push).
\end{abstract}
