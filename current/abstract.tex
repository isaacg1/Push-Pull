\begin{abstract}
This paper proves that push-pull block puzzles in 3D are PSpace-Complete to solve and push-pull block puzzles in 2D with thin walls are NP-Hard to solve. Push-pull block puzzles are a recreational motion planning problem, similar to Sokoban, that involves moving a `robot' on a grid with obstacles. The obstacles cannot be traversed by the robot, but some can be pushed or pulled by the robot into adjacent squares. Thin walls prevent movement between two adjacent squares. This work follows in a long line of algorithms and complexity work on similar problems \cite{PushPull91}\cite{Push100}\cite{Push*00}\cite{PushPushk04}\cite{non-crossing01}\cite{DO92}\cite{Push2F02}\cite{Sokoban98}\cite{DZ96}\cite{Pull10}. The 2D push-pull block puzzle shows up in The Legend of Zelda giving another proof of hardness for the game\cite{NintendoFun2014}. This variant of block pushing puzzles is of particular interest to the authors because it is fully reversible, meaning the inverse of any action that has been taken is valid.
\end{abstract}
%\xxx{address 2D portal related to push-pull? Also more games citing?}
